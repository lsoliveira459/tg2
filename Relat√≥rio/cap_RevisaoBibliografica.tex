%TCIDATA{LaTeXparent=0,0,relatorio.tex}
\ifx\compilewholereport\undefined
	\documentclass[11pt,a4paper,oneside]{book}
	
	% Escolher um dos seguintes formatos:
	\usepackage{ft2unb} % segue padrão de fontes do Latex
	
	% Pacotes
	\usepackage{graphicx}
	\usepackage{amsfonts}
	\usepackage{amsmath}
	\usepackage{amssymb}
	\usepackage[thmmarks,amsmath]{ntheorem}
	\usepackage{boxedminipage}
	\usepackage{theorem}
	\usepackage{fancybox}
	\usepackage{fancyhdr}
	\usepackage{url}
	\usepackage{afterpage}
	\usepackage{color}
	\usepackage{colortbl}
	\usepackage{rotating}
	\usepackage{makeidx}
	\usepackage{indentfirst}
	\usepackage{bibentry}
	\usepackage{subcaption}
	\usepackage{todonotes}
	\presetkeys{todonotes}{inline}{}
	
	\begin{document}
	\frontmatter
	\listoftodos
	\mainmatter
	
	%%%%%%%%%%%%%%%%%%%%%%%%%%%%
	%%%%%%%% Apagar coisas acima
	%%%%%%%%%%%%%%%%%%%%%%%%%%%%
\fi
                      
\chapter{Revisão Bibliográfica}\label{CapRevisaoBibliografica}

% Resumo opcional. Comentar se não usar.
\resumodocapitulo{Este capítulo visa apresentar as ferramentas disponíveis e o estado-da-arte da auto-reconfiguração.}

\section{Introdução}
\todo{Apresentar o estado-da-arte}

\section{Ferramentas}
\todo{ISE}
\todo{XST}
\todo{XPS}
\todo{SDK}
\todo{PlanAhead}
\todo{iMPACT}
\todo{ChipScope}

\section{Componentes}
\subsection{\textit{Intellectual Property}}
\todo{Clock}
\todo{Memória DDR3, MIG, MIS}
\subsection{Interfaces}
\todo{SelectMAP}
\todo{ICAP e ICAPE2}

\ifx\compilewholereport\undefined
	\bibliographystyle{authordate1} 
	%\bibliography{bibliografia}
	\newsavebox\mytempbib\savebox\mytempbib{\parbox{\textwidth}{\bibliography{bibliografia}}}

	\end{document}
\fi