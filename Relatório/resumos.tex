%TCIDATA{LaTeXparent=0,0,relatorio.tex}

\resumo{Resumo}{
A reconfiguração dinâmica de circuitos digitais representa o estado da arte em tecnologias de computação reconfigurável.
Através dela é possível reduzir o gasto energético, simplificar projetos, aumentar o desempenho geral do projeto e até resolver problemas antes impossíveis.
Este estudo buscou conhecer e apresentar as ferramentas e tecnologias necessárias para o desenvolvimento de um projeto deste tipo. Os cinco experimentos realizados também serviram para demonstrar e explicar o uso de alguns periféricos úteis, tais como memórias Flash e RAM, o HWICAP (necessário para a reconfiguração), a interface UART, dentre outros.
Neles, também foram descritos os processos necessários para o desenvolvimento de periféricos customizados, podendo, inclusive, serem reconfiguráveis.
Os resultados obtidos em cada experimento apresentam considerações importantes na opção pelo uso dessa tecnologia.
Conseguiu-se, por fim, definir um fluxo de projeto genérico que pode ser utilizado em grande parte dos projetos que necessitam de reconfiguração dinâmica.}

\paragraph{Palavras Chave:} Reconfiguração Dinâmica; Autorreconfiguração; Xilinx; MicroBlaze; DDR3; Periféricos.

\vspace*{2cm}

\resumo{Abstract}{
The dynamic reconfiguration of digital circuitry represents the state-of-art in reconfigurable computation.
By using it, it's possible to reduce the power, simplify projects, increase the overall system performance and even solve otherwise unsolvable problems.
This study tried to explore and present the tools and technologies required to develop a project that used dynamic reconfiguration.
The five experiments performed were also used to demonstrate and explain the use of a series of useful peripherals, such as Flash and RAM memories, the HWICAP peripheral, and the UART interface, and to describe the process required to build custom peripherals, including reconfigurable peripherals.
The results attained in each experiment are used to show important considerations when choosing this technology.
In the end, it was possible to define a generic project flow that can be used in most of the projects willing to use dynamic reconfiguration.}

\paragraph{Keywords:} Dynamic Reconfiguration; Self-reconfiguration; Xilinx; MicroBlaze; DDR3; Peripherals.