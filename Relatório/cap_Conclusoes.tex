%TCIDATA{LaTeXparent=0,0,relatorio.tex}
                      

\chapter{Conclus\~{o}es}\label{CapConclusoes}
No decorrer do desenvolvimento deste trabalho, pode-se observar a forma correta de se realizar projetos para reconfiguração dinâmica e autorrenconfiguração.
Observou-se as pecuriaridades e problemas de tais tipos de projetos, atentando para as formas corretas de resolvê-los.

O experimento 1 mostrou e explicou como realizar a reconfiguração dinâmica através de um computador.
Este experimento foi importante para comprovar o conceito e a capacidade do kit de suportar esta tecnologia.
O experimento 2 abordou o desenvolvimento de um microcontrolador MicroBlaze com periféricos para o controle de memórias QSPI Flash e DDR3.
Como foi explicado, memórias apresentam um papel extremamente importante na autorreconfiguração uma vez que nela a placa deve usar apenas recursos próprios para tudo.
O experimento 3 apresentou um estudo sobre o arquivo binário carregado pelas interfaces de reconfiguração, bem como o desenvolvimento de um programa capaz de interpretá-lo.
Este experimento permitiu ainda um maior entendimento sobre a inicialização da memória Flash para que ela fosse acessada em execuções subsequêntes.
O experimento 4, o único que não atingiu seu objetivo, foi uma tentiva de se construir um sistema que carregasse as configurações da memória Flash para a memória DDR3 para então transmití-la as interfaces de reconfiguração.
Diversos problemas surgiram com relação a memória DDR3, forçando a removê-la do sistema, e algumas dúvidas surgiram com relação a inicialização arquivo binário final com o programa embarcado.
No experimento 5, conseguiu-se solucionar o problema da inicialização do arquivo binário, obtendo-se ao final um sistema autenticamente autorreconfigurável.
O módulo reconfigurável, porém, fazia parte do microcontrolador, interagindo com ele através da interface AXI Lite.

Nota-se que prentende-se ainda, com os resultados alcaçados, desenvolver outros experimentos para continuar o estudo deste tema.
Com o auxílio do fluxo de projeto apresentado no capítulo de resultados, deseja-se modificar o experimento 5 para o seu funcionamento com o MicroBlaze apenas como um componentes, ao invés de \lq\lq{}Top\rq\rq{}.
Em seguida, pode-se continuar para a remoção completa do microprocessador, permitindo que o processo inteiro seja controlado apenas por máquinas de estado implementadas em \textit{hardware}.
Deseja-se ainda a correta integração da memória DDR3 ao projeto, removendo o gargalo no tempo de programação introduzido pela memória não-volatil.